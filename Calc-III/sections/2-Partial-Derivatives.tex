\section{Partial Derivatives}

%%%%%%%%%%%%%%%%%%%%%%%%%%%%%%%%%%%%%
%  First Order Partial Derivatives  %
%%%%%%%%%%%%%%%%%%%%%%%%%%%%%%%%%%%%%

\subsection{First Order Partial Derivatives}

\Definition{First Order Partial Derivative}{
    The \textbf{first order partial derivative} of a function $f(x,y)$ is the derivative of $f$ with respect to one variable while treating the other variable as a constant.
    The partial derivative of $f$ wrt $x$ is denoted by:
    \[ f_x(x,y) = \lim_{h \to 0} \frac{f(x+h,y) - f(x,y)}{h} \]
    And the partial derivative of $f$ wrt $y$ is denoted by:
    \[ f_y(x,y) = \lim_{h \to 0} \frac{f(x,y+h) - f(x,y)}{h} \]
    They can also be written in the following notations:
    \[ f_x(x,y) = f_x = \pdv{f}{x} = \pdv{}{x} \left(f(x,y)\right) = D_x f \]
    \[ f_y(x,y) = f_y = \pdv{f}{y} = \pdv{}{y} \left(f(x,y)\right) =  D_y f \]
}

%%%%%%%%%%%%%%%%%%%%%%%%%%%%%%%%%%%%%%%%%%%%
%  Interpretations of Partial Derivatives  %
%%%%%%%%%%%%%%%%%%%%%%%%%%%%%%%%%%%%%%%%%%%%

\subsection{Interpretations of Partial Derivatives}

Much like the first derivative of a function of one variable, the first order partial derivatives of a function of multiple variables can be interpreted as the slope of the tangent line to the surface defined by the function at a point.

\Note[Slopes of Traces]{
    Partial derivatives are the slopes of traces. The partial derivative $f_x(a,b)$ is the slope of the trace of $f(x,y)$ for the plane $y=b$ at the point $(a,b)$. Likewise, the partial derivative $f_y(x,y)$ is the slope of the trace of $f(x,y)$ for the plane $x=a$ at the point $(a,b)$
}

\Example{Determine if $f(x,y) = \frac{x^2}{y^3}$ is increasing or decreasing at $(2,5)$, if: \\
    \quad (a) we allow $x$ to vary and hold $y$ fixed, \\
    \quad (b) we allow $y$ to vary and hold $x$ fixed. \\
}{
    \textbf{(a)} To find the partial derivative with respect to $x$, we treat $y$ as a constant: \[ 
        f_x(x,y) = \frac{2x}{y^3} \qquad \implies \qquad f_x(2,5) = \frac{4}{125} > 0
    \]
    This means that $f$ is increasing in the $x$ direction at the point $(2,5)$. \\
    \textbf{(b)} To find the partial derivative with respect to $y$, we treat $x$ as a constant: \[ 
        f_y(x,y) = -\frac{3x^2}{y^4} \qquad \implies \qquad f_y(2,5) = -\frac{12}{625} < 0 
    \]
    This means that $f$ is decreasing in the $y$ direction at the point $(2,5)$.
}

Partial derivatives can also be interpreted as the slope of the tangent plane to the surface defined by the function at a point. The tangent plane is a linear approximation of the surface at that point.

\Example{Find the slopes of the traces to $z = 10 - 4x^2 - y^2$ at the point $(1,2)$.}{
    The partial derivative with respect to $x$ is:
    \[
        f_x(x,y) = -8x \qquad \implies \qquad f_x(1,2) = -8
    \]
    The partial derivative with respect to $y$ is:
    \[
        f_y(x,y) = -2y \qquad \implies \qquad f_y(1,2) = -4
    \]
    Thus, the slope of the trace in the $x$ direction at $(1,2)$ is $-8$, and the slope of the trace in the $y$ direction at $(1,2)$ is $-4$.
}

We can also use partial derivatives to find the equations of the tangent lines to the traces of a surface at a point.

\Example{Find the vector equations of the tangent lines to the traces to $z = 10 - 4x^2 - y^2$ at the point $(1,2)$}{
    The point on the trace is \[ 
        (1, 2, f(1,2)) = (1, 2, 10-4(1)^2 - (2)^2) = (1, 2, 2)
    \]
    Hence, the equation of the tangent line to the trace for the plane $y=2$ is: \[ 
        \vec{r_x}(t) = \langle 1,2,2 \rangle + t \langle 1,0,-8 \rangle = \langle 1+t, 2, 2-8t \rangle
    \]
    And the equation of the tangent line to the trace for the plane $x=1$ is: \[ 
        \vec{r_y}(t) = \langle 1,2,2 \rangle + t \langle 0,1,-4 \rangle = \langle 1, 2+t, 2-4t \rangle
    \]
}

\Example{Find the vector equations of the tangent lines to the traces for $f(x,y) = \sin x \cos y$ at $\left( \frac{\pi}{3}, \frac{-\pi}{4} \right)$}{
    The point on the trace is \[ 
        \left( \frac{\pi}{3}, - \frac{\pi}{4}, f \left( \frac{\pi}{3}, - \frac{\pi}{4} \right) \right) = \left( \frac{\pi}{3}, - \frac{\pi}{4}, \sin \left( \frac{\pi}{3} \right) \cos \left( - \frac{\pi}{4} \right) \right) = \left( \frac{\pi}{3}, - \frac{\pi}{4}, \frac{\sqrt{6}}{4} \right)
    \]
    Hence, the equation of the tangent line to the trace for the plane $y = -\frac{\pi}{4}$ is:
    \begin{align*}
        \vec{r_x}(t) &= \Biggl\langle \frac{\pi}{3}, - \frac{\pi}{4}, \frac{\sqrt{6}}{4} \Biggr\rangle + t \Biggl\langle 1, 0, f_x(x,y) \Biggr\rangle \\
        &= \Biggl\langle \frac{\pi}{3}, - \frac{\pi}{4}, \frac{\sqrt{6}}{4} \Biggr\rangle + t \Biggl\langle 1, 0, \cos \left( \frac{\pi}{3} \right) \cos \left( - \frac{\pi}{4} \right) \Biggr\rangle \\
        &= \Biggl\langle \frac{\pi}{3}, - \frac{\pi}{4}, \frac{\sqrt{6}}{4} \Biggr\rangle + t \Biggl\langle 1, 0, \frac{1}{2\sqrt{2}} \Biggr\rangle
    \end{align*}
    And the equation of the tangent line to the trace for the plane $x = \frac{\pi}{3}$ is: 
    \begin{align*}
        \vec{r_y}(t) &= \Biggl\langle \frac{\pi}{3}, - \frac{\pi}{4}, \frac{\sqrt{6}}{4} \Biggr\rangle + t \Biggl\langle 0, 1, f_y(x,y) \Biggr\rangle \\
           &= \Biggl\langle \frac{\pi}{3}, - \frac{\pi}{4}, \frac{\sqrt{6}}{4} \Biggr\rangle + t \Biggl\langle 0, 1, - \sin \left( \frac{\pi}{3} \right) \sin \left( - \frac{\pi}{4} \right) \Biggr\rangle \\
           &= \Biggl\langle \frac{\pi}{3}, - \frac{\pi}{4}, \frac{\sqrt{6}}{4} \Biggr\rangle + t \Biggl\langle 0, 1, \frac{\sqrt{6}}{4} \Biggr\rangle
    \end{align*}
}

%%%%%%%%%%%%%%%%%%%%%%%%%%%%%%%%%%%%%%
%  Higher Order Partial Derivatives  %
%%%%%%%%%%%%%%%%%%%%%%%%%%%%%%%%%%%%%%

\subsection{Higher Order Partial Derivatives}
\Note[Second Order Partial Derivatives]{
    The \textbf{Second order partial derivatives} of a function $f(x,y)$ are the partial derivatives of the first order partial derivatives. 
    The second order partial derivatives are denoted by:
    \[ (f_x)_x = f_{x x} = \pdv{}{x} \left( \pdv{f}{x} \right) = \pdv[2]{f}{x} \]
    \[ (f_x)_y = f_{x y} = \pdv{}{y} \left( \pdv{f}{x} \right) = \pdv[2]{f}{y}{x} \]
    \[ (f_y)_x = f_{y x} = \pdv{}{x} \left( \pdv{f}{y} \right) = \pdv[2]{f}{x}{y} \]
    \[ (f_y)_y = f_{y y} = \pdv{}{y} \left( \pdv{f}{y} \right) = \pdv[2]{f}{y} \]
}

\Note[Clairaut's Theorem]{
    If the second order partial derivatives $f_{x y}$ and $f_{y x}$ are continuous at a point, then they are equal at that point:
    \[ f_{x y} = f_{y x} \]
}

Like second order derivatives, there are higher order partial derivatives as well. The third order partial derivatives are denoted by:
\[ f_{x x x} = \pdv[3]{f}{x} \]
\[ f_{x y x} = (f_{xy})_x = \pdv{}{x} \left( \pdv{f}{y,x} \right) = \pdv{f}{x,y,x} \]
\[ f_{y x x} = (f_{yx})_x = \pdv{}{x} \left( \pdv{f}{x,y} \right) = \pdv{f}{x^2,y} \]
This also applies to functions of more than two variables. For example, \[ 
    f_{xz}(x,y,z) = f_{zx}(x,y,z)
\]

\Note[Extension of Clairaut's Theorem]{
    In general, we can extend Clairaut's theorem to any function and mixed partial derivatives. That means: \[ 
        f_{s s r t s r r} = f_{t r s r s s r} = f_{r r s s s t} = \dots
     \]
}

\Example{Find all the second order partial derivatives of function $Q(u,v,w) = u^4 \sin w^2 - \frac{2v}{u^4} + \ln(v^2w)$}{
    To find the second order partial derivatives, we first find the first order partial derivatives:
    \begin{align*}
        Q_u &= 4u^3 \sin w^2 + \frac{8v}{u^5} \\
        Q_v &= \frac{-2}{u^4} + \frac{2vw}{v^2w} = - \frac{2}{u^4} + \frac{2}{v} \\
        Q_w &= 2u^4 w \cos w^2 + \frac{1}{w}
    \end{align*}
    Now we can find the second order partial derivatives:
    \begin{align*}
        Q_{uu} &= 12u^2 \sin w^2 - \frac{40v}{u^6} \\
        Q_{uv} &= Q_{vu} = \frac{8}{u^5} \\
        Q_{uw} &= 8u^3 w \cos w^2 \\
        Q_{vv} &= - \frac{2}{v^2} \\
        Q_{vw} &= Q_{wv} = 0 \\
        Q_{ww} &= 2u^4 \cos w^2 - 4u^4 w^2 \sin w^2 - \frac{1}{w^2}
    \end{align*}
}

\Example{Given $w = \ln\left(\frac{xy}{z}\right) + 8x^4y^3\sqrt{z}, find \pdv[1,2,1,1]{w}{x,z,y,x}$}{
    Using Clairaut's theorem, \[ 
        \pdv[1,2,1,1]{w}{x,z,y,x} = \pdv[2,1,2]{w}{x,y,z}
    \]
    Now,
    \begin{align*}
        \pdv{w}{x} &= \frac{y}{z} \cdot \frac{z}{xy} + 32x^3y^3\sqrt{z} = \frac{1}{x} + 32x^3y^3\sqrt{z} \\
        \pdv[2]{w}{x} &= - \frac{1}{x^2} + 96x^2y^3\sqrt{z} \\
        \pdv[2,1]{w}{x,y} &= 288x^2y^2\sqrt{z} \\
        \pdv[2,1,1]{w}{x,y,z} &= \frac{144x^2y^2}{\sqrt{z}} \\
        \pdv[2,1,2]{w}{x,y,z} &= -72x^2y^2z^{-3/2}
    \end{align*}
}

\Example{Given $f(x,y) = \frac{x^6}{1+6y} - \cos(x^2) + 6e^{x}\sin(y)$, find $f_{xxyxyx}$}{
    \begin{align*}
        f_x &= \frac{6x^5}{1+6y} + 2x\sin(x^2) + 6e^{x}\sin(y) \\
        f_{xx} &= \frac{30x^4}{1+6y} + 2\sin(x^2) + 4x^2\cos(x^2) + 6e^{x}\sin(y) \\ 
        f_{xxx} &= \frac{120x^3}{1+6y} + 12x \cos(x^2) - 8x^3 \sin(x^2) + 6e^{x}\sin(y) \\
        f_{xxxx} &= \frac{360x^2}{1+6y} + 12 \cos(x^2) - 48x^2 \sin(x^2) - 16x^4 \cos(x^2) + 6e^{x}\sin(y) \\
        f_{xxxxy} &= - \frac{2160x^2}{(1+6y)^2} + 6e^{x}\cos(y) \\
        f_{xxxxyy} &= \frac{25920x^2}{(1+6y)^3} - 6e^{x}\sin(y)
    \end{align*}
}

%%%%%%%%%%%%%%%%%%%
%  Differentials  %
%%%%%%%%%%%%%%%%%%%

\subsection{Differentials}

\Note[Differentials]{
    The \textbf{differential} of a function $f(x,y)$ is a linear approximation of the change in the function at a point. \\
    The differential of $f$ is denoted by:
    \[ \dd{f} = f_x \dd{x} + f_y \dd{y} \]
    where $dx$ and $dy$ are small changes in $x$ and $y$, respectively. \\
    For a given function $w = g(x,y,z)$, the differential is given by:
    \[ \dd{w} = g_x \dd{x} + g_y \dd{y} + g_z \dd{z} \]
}

\Example{Compute the differential for $u = \frac{t^3r^6}{s^2}$}{
    \[ 
        \dd{u} = \frac{3t^2r^6}{s^2} \dd{t} + \frac{6t^3r^5}{s^2} \dd{r} - \frac{2t^3r^6}{s^3} \dd{s}
    \]
}

%%%%%%%%%%%%%%%%
%  Chain Rule  %
%%%%%%%%%%%%%%%%

\subsection{Chain Rule}

\underline{\textbf{Case 1:}}
    If $z = f(x,y)$, $x = g(t)$, and $y = h(t)$, then the chain rule states that:
    \[ \dv{z}{t} = \pdv{f}{x} \dv{x}{t} + \pdv{f}{y} \dv{y}{t} \]
    \[ \text{Or,} \quad \dv{z}{t} = f_x \dv{x}{t} + f_y \dv{y}{t} \]

\Example{Compute $\dv{z}{t}$ for $z=xe^{xy}$, $x=t^2$, $y=t^{-1}$}{
    \begin{align*}
        \dv{z}{t} &= \pdv{f}{x} \dv{x}{t} + \pdv{f}{y} \dv{y}{t} \\
        &= (e^{xy} + yxe^{xy}) (2t) + x^2e^{xy} (-t^{-2}) \\
        &= 2t (e^{xy} + xy e^{xy}) - x^2e^{xy}t^{-2} \\
        &= 2t (e^t + te^t) - t^4e^t t^{-2} \\ 
        &= 2t e^{t} + t^2 e^{t}
    \end{align*}
}

\underline{\textbf{Case 2:}}
    If $z = f(x,y)$, $x = g(s,t)$, and $y = h(s,t)$, then the chain rule states that:
    \[ \pdv{z}{s} = \pdv{f}{x} \pdv{x}{s} + \pdv{f}{y} \pdv{y}{s} \]
    \[ \text{Or,} \quad \pdv{z}{s} = f_x \pdv{x}{s} + f_y \pdv{y}{s} \]
    \[ \pdv{z}{t} = \pdv{f}{x} \pdv{x}{t} + \pdv{f}{y} \pdv{y}{t} \]
    \[ \text{Or,} \quad \pdv{z}{t} = f_x \pdv{x}{t} + f_y \pdv{y}{t} \]

\Example{Find $\pdv{z}{s}$ and $\pdv{z}{t}$ for $z=e^{2r}\sin(3\theta)$, $r=st-t^2$, $\theta=\sqrt{s^2+t^2}$}{
    \begin{align*}
        \pdv{z}{s} &= \pdv{f}{r} \pdv{r}{s} + \pdv{f}{\theta} \pdv{\theta}{s} \\
        &= (2e^{2r}\sin(3\theta)) (t + 0) + (3e^{2r}\cos(3\theta)) \left(\frac{s}{\sqrt{s^2+t^2}}\right) \\
        &= 2te^{2r}\sin(3\theta) + 3e^{2r}\cos(3\theta) \frac{s}{\sqrt{s^2+t^2}} \\ 
        &= 2t e^{2(st-t^2)}\sin(3\sqrt{s^2+t^2}) + 3e^{2(st-t^2)}\cos(3\sqrt{s^2+t^2}) \frac{s}{\sqrt{s^2+t^2}}
    \end{align*}
    And,
    \begin{align*}
        \pdv{z}{t} &= \pdv{f}{r} \pdv{r}{t} + \pdv{f}{\theta} \pdv{\theta}{t} \\
        &= (2e^{2r}\sin(3\theta)) (s-2t) + (3e^{2r}\cos(3\theta)) \left(\frac{t}{\sqrt{s^2+t^2}}\right) \\
        &= 2(s-2t)e^{2r}\sin(3\theta) + 3e^{2r}\cos(3\theta) \frac{t}{\sqrt{s^2+t^2}} \\ 
        &= 2(s-2t) e^{2(st-t^2)}\sin(3\sqrt{s^2+t^2}) + 3e^{2(st-t^2)}\cos(3\sqrt{s^2+t^2}) \frac{t}{\sqrt{s^2+t^2}}
    \end{align*}
}

\Note[Chain Rule]{
    Given the following conditions:
    \begin{enumerate}[label=(\roman*)]
        \item $z = f(x_1, x_2, \dots, x_n)$ is a function of $n$ variables,
        \item Each variable $x_i(t_1, t_2, \dots, t_m)$ is a function of $m$ variables,
    \end{enumerate}
    Then for any variable $t_i$ $(i = 1,2,\dots,m)$, we have the following chain rule:
    \[ 
        \boxed{ \pdv{z}{t_i} = \pdv{z}{x_1} \pdv{x_1}{t_i} + \pdv{z}{x_2} \pdv{x_2}{t_i} + \cdots + \pdv{z}{x_n} \pdv{x_n}{t_i} }
    \]
}

\Example{Compute $\pdv[2]{f}{\theta}$ for $f(x,y)$ if $x=r\cos\theta$ and $y=r\sin\theta$}{
    \begin{align*}
        \pdv{f}{\theta} &= \pdv{f}{x}\pdv{x}{\theta} + \pdv{f}{y}\pdv{y}{\theta} \\
        &= -r \sin\theta \pdv{f}{x} + r \cos\theta \pdv{f}{y}
    \end{align*}
    Now, we know the second derivative is \[ 
        \pdv[2]{f}{\theta} = \pdv{}{\theta} \left( \pdv{f}{\theta} \right) = \pdv{}{\theta} \left( -r \sin\theta \pdv{f}{x} + r \cos\theta \pdv{f}{y} \right) \\ 
    \]
    Now, we can separately compute $\pdv{}{\theta} \left( \pdv{f}{x} \right)$ and $\pdv{}{\theta} \left( \pdv{f}{y} \right)$:
    \begin{align*}
        \pdv{}{\theta} \left( \pdv{f}{x} \right) &= -r \sin\theta \pdv{}{x} \left( \pdv{f}{x} \right) + r \cos\theta \pdv{}{y} \left( \pdv{f}{x} \right) \\
            &= -r \sin\theta \pdv[2]{f}{x} + r \cos\theta \pdv{f}{y,x} \\ 
        \pdv{}{\theta} \left( \pdv{f}{y} \right) &= -r \sin\theta \pdv{}{x} \left( \pdv{f}{y} \right) + r \cos\theta + \pdv{}{y} \left( \pdv{f}{y} \right) \\
            &= -r \sin\theta \pdv{f}{x,y} + r \cos\theta \pdv[2]{f}{y}
    \end{align*}
    Finally, we can substitute these into the second derivative:
    \begin{align*}
        \pdv[2]{f}{\theta} &= \pdv{}{\theta} \left( -r \sin\theta \pdv{f}{x} + r \cos\theta \pdv{f}{y} \right) \\
        &= -r \cos\theta \pdv{f}{x} - r \sin\theta \left( -r \sin\theta \pdv[2]{f}{x} + r \cos\theta \pdv{f}{y,x} \right) \\
        &\quad - r \sin\theta \pdv{f}{y} + r \cos\theta \left( -r \sin\theta \pdv{f}{x,y} + r \cos\theta \pdv[2]{f}{y} \right) \\
        &= -r \cos\theta \pdv{f}{x} + r^2 \sin^2\theta \pdv[2]{f}{x} - r^2 \sin\theta \cos\theta \pdv{f}{y,x} \\
        &\quad - r \sin\theta \pdv{f}{y} - r^2 \sin\theta \cos\theta \pdv{f}{x,y} + r^2 \cos^2\theta \pdv[2]{f}{y} \\ 
        &= - r \left( \cos\theta \pdv{f}{x} + \sin\theta \pdv{f}{y} \right) \\ 
        &\quad + r^2 \left( \sin^2\theta \pdv[2]{f}{x} - 2 \sin\theta \cos\theta \pdv{f}{x,y} + \cos^2\theta \pdv[2]{f}{y} \right)
    \end{align*}
}

\subsubsection{Implicit Differentiation}

\Note[Implicit Differentiation]{
    If $F(x,y)=0$ is a function where $y=y(x)$, then we can use implicit differentiation to find the derivative of $y$ with respect to $x$. The chain rule gives us:
    \[ 
        F_x + F_y \frac{dy}{dx} = 0 \quad \implies \quad \frac{dy}{dx} = -\frac{F_x}{F_y}
    \]
    This can be extended to functions of more than two variables. We can start by assuming that $z=f(x,y)$ and we want to find $\pdv{z}{x}$ and $\pdv{z}{y}$. \\ 
    To find $\pdv{z}{x}$, we differentiate both sides wrt $x$: \[ 
        \pdv{F}{x} \pdv{x}{x} + \pdv{F}{y} \pdv{y}{x} + \pdv{F}{z} \pdv{z}{x} = 0
    \]
    Since $\pdv{x}{x} = 1$ and $\pdv{y}{x} = 0$, we get:
    \[ 
        \pdv{z}{x} = - \frac{F_x}{F_z} \quad \text{ and } \quad 
        \pdv{z}{y} = - \frac{F_y}{F_z}
    \]
}

\Example{Find $\dv{y}{x}$ for $x \cos(3y) + x^3y^5 = 3x - e^{xy}$}{
    First, we rearrange the equation in the form $F(x,y) = 0$: \[ 
        x \cos(3y) + x^3y^5 - 3x + e^{xy} = 0 
    \]
    Now, the derivative is: \[ 
        \dv{y}{x} = - \frac{F_x}{F_y} = - \frac{\cos(3y) + 3x^2y^5 - 3 + ye^{xy}}{-3x \sin(3y) + 5x^3y^4 + xe^{xy}}
    \]
}

\Example{Find $\pdv{z}{x}$ and $\pdv{z}{y}$ for $x^2 \sin(2y-5z) = 1 + y \cos(6zx)$}{
    First, let's rearrange the equation in the form $F(x,y,z) = 0$: \[ 
        x^2 \sin(2y-5z) - 1 - y \cos(6zx) = 0
    \]
    Now, the derivatives are:
    \begin{align*}
        \pdv{z}{x} &= - \frac{F_x}{F_z} = - \frac{2x \sin(2y-5z) + 6yz \sin(6zx)}{-5x^2 \cos(2y-5z) + 6xy \sin(6zx)} \\
        \pdv{z}{y} &= - \frac{F_y}{F_z} = - \frac{2x^2 \cos(2y-5z) - \cos(6zx)}{-5x^2 \cos(2y-5z) + 6xy \sin(6zx)}
    \end{align*}
}

%%%%%%%%%%%%%%%%%%%%%%%%%%%%%
%  Directional Derivatives  %
%%%%%%%%%%%%%%%%%%%%%%%%%%%%%

\subsection{Directional Derivatives}

\Definition{Directional Derivative}{
    The rate of change of $f(x,y)$ in the direction of the unit vector $\vec{u} = \langle a,b \rangle$ is called the \textbf{directional derivative} of $f$ and is denoted by $D_{\vec{u}} f(x,y)$. The definition of the directional derivative is:
    \[ 
        D_{\vec{u}} f(x,y) = \lim_{h \to 0} \frac{f(x+ah,y+bh) - f(x,y)}{h}
    \]
}

Now, in practice, finding this limit can be difficult. We can derive an equivalent formula for taking directional derivatives.

Let's define a new function of one variable: \[ 
    g(z) = f(x_0+az, y_0+bz)
\]
where $x_0$, $y_0$, $a$, $b$ are constants. Then, by the definition of the derivative, we have \[ 
    g'(z) = \lim_{h \to 0} \frac{g(z+h) - g(z)}{h}
\]
For $z=0$, we have: \[
    g'(0) = \lim_{h \to 0} \frac{g(h)-g(0)}{h} = \lim_{h \to 0} \frac{f(x_0+ah, y_0+bh) - f(x_0,y_0)}{h} = D_{\vec{u}} f(x_0,y_0)
\]
Thus, we have the following relationship: \[ 
    g'(0) = D_{\vec{u} f(x_0,y_0)}
\]

Now, let's rewrite $g(z)$ as follows: \[ 
    g(z) = f(x,y) \quad \text{where } x = x_0 + az \text{ and } y = y_0 + bz
\]
We can now apply the chain rule to find $g'(z)$: \[ 
    g'(z) = \dv{g}{z} = \pdv{f}{x} \dv{x}{z} + \pdv{f}{y} \dv{y}{z} = f_x(x,y) a + f_x(x,y) b
\]

If we take $z=0$, we get $x=x_0$ and $y=y_0$, and finally we have: \[ 
    D_{\vec{u}} f(x_0,y_0) = g'(0) = f_x(x_0,y_0) a + f_y(x_0,y_0) b
\]

\Note{
    \[ D_{\vec{u}} f(x,y) = f_x(x,y) a + f_y(x,y) b \]
    \[ D_{\vec{u}} f(x,y,z) = f_x(x,y,z)a + f_y(x,y,z)b + f_z(x,y,z)c \]
}

\Example{Find each of the directional derivatives: \\ 
    \quad (a) $D_{\vec{u}}f(8,1,2)$ where $f(x,y,z) = \ln \frac{x}{z} + \ln \frac{z}{y} + xy^2$ in the direction of $\vec{v} = \langle 1,5,2 \rangle$ \\ 
    \quad (b) $D_{\vec{u}} f(x,y)$ where $f(x,y)=xe^{xy}+y$ and $\vec{u}$ is the unit vector in the direction of $\theta=\frac{2\pi}{3}$
}{
    \textbf{(a)} First, the unit vector in the direction of $\vec{v}$ is: \[ 
        \vec{u} = \frac{\vec{v}}{||\vec{v}||} = \frac{\langle 1,5,2 \rangle}{\sqrt{1^2+5^2+2^2}} = \frac{\langle 1,5,2 \rangle}{\sqrt{30}}
    \]
    Simplifying the function, we have: \[ 
        f(x,y,z) = \ln(x) - \ln(z) + \ln(z) - \ln(y) + xy^2 = \ln(x) - \ln(y) + xy^2
    \]
    The directional derivative is then,
    \begin{align*}
        D_{\vec{u}}f(x,y,z) &= \frac{1}{\sqrt{30}} \left[ f_x(x,y,z) + 5f_y(x,y,z) + 2f_z(x,y,z) \right] \\
        D_{\vec{u}}f(8,1,2) &= \frac{1}{\sqrt{30}} \left[ 1 \left( \frac{1}{x} + y^2 \right) + 5 \left( -\frac{1}{y} + 2xy \right) + 2\cdot0 \right]_{(8,1,2)} \\
        &= \frac{1}{\sqrt{30}} \left( \frac{1}{8} + 1 - 5 + 80 \right) \\ 
        &= \frac{609}{8 \sqrt{30}}
    \end{align*}
    \textbf{(b)} The unit vector in the direction of $\theta = \frac{2\pi}{3}$ is: \[ 
        \vec{u} = \Biggl\langle \cos( \frac{2\pi}{3} ), \sin( \frac{2\pi}{3} ) \Biggr\rangle = \Biggl\langle - \frac{1}{2}, \frac{\sqrt{3}}{2} \Biggr\rangle
    \]
    So, the directional derivative is:
    \begin{align*}
        D_{\vec{u}} f(x,y) &= \left( - \frac{1}{2} \right) \left( e^{xy} + xy e^{xy} \right) + \left( \frac{\sqrt{3}}{2} \right) \left( x^2 e^{xy} + 1 \right) \\
        D_{\vec{u}} f(2,0) &= \left( - \frac{1}{2} \right) (1) + \left( \frac{\sqrt{3}}{2} \right) (5) \\ 
           &= \frac{5 \sqrt{3} - 1}{2}
    \end{align*}
}

Notice, the directional derivative can also be written in the following way:
\begin{align*}
    D_{\vec{u}}f(x,y,z) &= f_x(x,y,z)a + f_y(x,y,z)b + f_z(x,y,z) \\
        &= \langle f_x,f_y,f_z \rangle \cdot \langle a,b,c \rangle 
\end{align*}

In other words, the directional derivative is the dot product of the gradient vector and the unit vector in the direction of interest.

\Note[Gradient Vector]{
    The \textbf{gradient vector} of a function $f(x,y)$ is denoted by $\nabla f$ and is defined as:
    \[ 
        \nabla f = \langle f_x, f_y \rangle = \Biggl\langle \pdv{f}{x}, \pdv{f}{y} \Biggr\rangle
    \]
    For a function $f(x,y,z)$, the gradient vector is:
    \[ 
        \nabla f = \langle f_x, f_y, f_z \rangle = \Biggl\langle \pdv{f}{x}, \pdv{f}{y}, \pdv{f}{z} \Biggr\rangle
    \]
}

\Note[Directional Derivative]{
    The directional derivative can also be expressed in terms of the gradient vector: \[ 
        D_{\vec{u}} f(\vec{x}) = \nabla f \cdot \vec{u}
    \]
    where $\vec{x} = \langle x,y,z \rangle$ or $\vec{x} = \langle x,y \rangle$ depending on the function and $\vec{u}$ is the unit vector in the direction of interest. \\
}

\Example{Find the directional derivative \[ 
    D_{\vec{u}}f(\vec{x}) \text{ for } f(x,y,z) = \sin(yz) + \ln(x^2) \text{ at } (1,1,\pi) \] in the direction of $\vec{v} = \langle 1,1,-1 \rangle$
}{
    The gradient vector is: 
    \begin{align*}
        \nabla f(x,y,z) &= \Biggl\langle \frac{2}{x}, z \cos(yz), y \cos(yz) \Biggr\rangle \\
        \nabla f(1,1,\pi) &= \Biggl\langle \frac{2}{1}, \pi \cos(\pi), \cos(\pi) \Biggr\rangle = \langle 2, -\pi, -1 \rangle
    \end{align*}
    The unit vector in the direction of $\vec{v}$ is: \[ 
        \vec{u} = \frac{\vec{v}}{||\vec{v}||} = \frac{\langle 1,1,-1 \rangle}{\sqrt{3}} = \Biggl\langle \frac{1}{\sqrt{3}}, \frac{1}{\sqrt{3}}, -\frac{1}{\sqrt{3}} \Biggr\rangle
    \]
    Hence, the directional derivative is:
    \begin{align*}
        D_{\vec{u}}f(1,1,\pi) &= \nabla f \cdot \vec{u} \\
        &= \Biggl\langle 2, -\pi, -1 \Biggr\rangle \cdot \Biggl\langle \frac{1}{\sqrt{3}}, \frac{1}{\sqrt{3}}, -\frac{1}{\sqrt{3}} \Biggr\rangle \\
        &= 2 \cdot \frac{1}{\sqrt{3}} - \pi \cdot \frac{1}{\sqrt{3}} + 1 \cdot \frac{1}{\sqrt{3}} = \frac{3 - \pi}{\sqrt{3}}
    \end{align*}
}

\Theorem{}{
    The maximum value of $D_{\vec{u}}f(\vec{x})$ (and hence then the maximum rate of change of the function $f(\vec{x})$) is given by $||\nabla f(\vec{x})||$ and will occur in the direction given by $\nabla f(\vec{x})$.
}

\underline{\textbf{Proof:}} \\
We can use a nice fact about dot products as well as the fact that $\vec{u}$ is a unit vector to proof this theorem:
\[ D_{\vec{u}}f = \nabla f \cdot \vec{u} = ||\nabla f|| \; ||\vec{u}|| \cos\theta = ||\nabla f|| \cos\theta \]
where $\theta$ is the angle between the gradient and $\vec{u}$. \\
Now, the largest possible value of $\cos\theta$ is $1$, which occurs at $\theta=0$. Therefore, the maximum value of $D_{\vec{u}} f(\vec{x})$ is $||\nabla f(\vec{x})||$. Also, the maximum value occurs when the angle between the gradient and $\vec{u}$ is zero, or in other words, when $\vec{u}$ is pointing in the same direction as the gradient.

\Note{
    The gradient vector $\nabla f(x_0,y_0)$ is orthogonal (or perpendicular) to the level curve/contour curve $f(x,y)=k$ at the point $(x_0,y_0)$. Likewise, the gradient vector $\nabla f(x_0,y_0,z_0)$ is orthogonal to the level surface $f(x,y,z)=k$ at the point $(x_0,y_0,z_0)$.
}

\underline{\textbf{Proof:}} \\
Let $S$ be the level surface given by $f(x,y,z)=k$ and let $P(x_0,y_0,z_0)$ be a point on the surface $S$.

Now, let $C$ be any curve on the surface $S$ that contains the point $P$. Let $\vec{r}(t) = \langle x(t), y(t), z(t) \rangle$ be the vector equation for $C$ and suppose that $t_0$ is the value of $t$ such that $\vec{r}(t_0) = \langle x_0,y_0,z_0 \rangle$. In other words, $t_0$ is the value of $t$ that gives $P$.

Since $C$ lies on $S$, we know that the points on $C$ must satisfy the equation for $S$. That is \[ 
    f(x(t),y(t),z(t)) = k
\]
Using the chain rule, we get: \[ 
    \pdv{f}{x} \dv{x}{t} + \pdv{f}{y} \dv{y}{t} + \pdv{f}{z} \dv{z}{t} = 0
\]
Notice that $\nabla f = \langle f_x,f_y,f_z \rangle$ and $\vec{r}'(t) = \langle x'(t),y'(t),z'(t) \rangle$ so this becomes: \[ 
    \nabla f \cdot \vec{r}'(t) = 0
\]
At $t = t_0$, \[ 
    \nabla f(x_0,y_0,z_0) \cdot \vec{r}'(t_0) = 0
\]
This then tells us that the gradient vector at $P$ (i.e. $\nabla f(x_0,y_0,z_0)$) is orthogonal to the tangent vector $\vec{r}'(t_0)$ to any curve $C$ that passes through $P$ and on the surface $S$ and su must also be orthogonal to the surface S.
