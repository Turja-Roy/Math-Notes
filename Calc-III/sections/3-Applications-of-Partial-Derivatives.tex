\section{Applications of Partial Derivatives}

%%%%%%%%%%%%%%%%%%%%%%%%%%%%%%%%%%%%%%%%%%%%%%
%  Tangent Planes and Linear Approximations  %
%%%%%%%%%%%%%%%%%%%%%%%%%%%%%%%%%%%%%%%%%%%%%%

\subsection{Tangent Planes and Linear Approximations}

Let there be a point $(x_0,y_0)$ on function $z=f(x,y)$ in $\R^3$. Let $C_1$ be the trace to $f(x,y)$ for plane $y=y_0$ and $C_2$ be the trace to $f(x,y)$ for plane $x=x_0$. Now, let $L_1$ and $L_2$ be the tangent lines to $C_1$ and $C_2$ at point $(x_0,y_0)$, respectively. The tangent plane to $f(x,y)$ at point $(x_0,y_0)$ is defined as the plane that contains both lines $L_1$ and $L_2$.

\begin{figure}[htpb]
    \centering
    \includegraphics[width=0.6\textwidth]{3.1.1.png}
    \caption{Tangent plane}
\end{figure}

Now, we need to find the equation of the tangent plane. The general equation of a plane is given by \[ 
    a(x-x_0) + v(y-y_0) + c(z-z_0) = 0
\]
where $(x_0,y_0,z_0)$ is a point on the plane and $a$, $b$, and $c$ are the coefficients of the plane. The equation can be rewritten as \[ 
    z-z_0 = - \frac{a}{c}(x-x_0) - \frac{b}{c}(y-y_0)
\]
Let's rename the constants as follows: \[ 
    A = -\frac{a}{c}, \qquad B = -\frac{b}{c}
\]
Thus we get \[ 
    z-z_0 = A(x-x_0) + B(y-y_0)
\]

Now, assuming $y=y_0$ (i.e., $y$ is fixed) and $x=x_0$ (i.e., $x$ is fixed) we get respectively
\[ 
    z-z_0 = A(x-x_0) \qquad \text{and} \qquad z-z_0 = B(y-y_0)
\]
Note, these are the equations for the tangent lines $L_1$ and $L_2$ respectively, where the slopes are respectively $A=f_x(x_0,y_0)$ and $B=f_y(x_0,y_0)$. We also know $z_0=f(x_0,y_0)$.

Hence, the equation of the tangent plane is given by
\[ 
    \boxed{ z = f(x_0,y_0) + f_x(x_0,y_0)(x-x_0) + f_y(x_0,y_0)(y-y_0) }
\]

The linear approximation for a surface "near" the point $(x_0,y_0)$ is given then \[ 
    L(x,y) = f(x_0,y_0) + f_x(x_0,y_0)(x-x_0) + f_y(x_0,y_0)(y-y_0)
\]

\Example{Find the linear approximation to $z = \frac{10x^2}{x-y}$ at $(4,-1)$}{
    The linear approximation is given by \[ 
        L(x,y) = z(4,-1) + z_x(4,-1)(x-4) + z_y(4,-1)(y+1)
    \]
    Here,
    \begin{align*}
        z(4,-1) &= \frac{10 \times 4^2}{4 - (-1)} \\
                &= \frac{160}{5} = 32 \\
        z_x(4,-1) &= \frac{\partial}{\partial x} \left( \frac{10x^2}{x-y} \right) \bigg|_{(4,-1)} \\
                &= \frac{20x(x-y) - 10x^2}{(x-y)^2} \bigg|_{(4,-1)} \\
                &= \frac{20 \times 4 (4 - (-1)) - 10 \times 4^2}{(4 - (-1))^2} \\
                &= \frac{80 \times 5 - 160}{25} = \frac{400 - 160}{25} = \frac{240}{25} = 9.6 \\
        z_y(4,-1) &= \frac{\partial}{\partial y} \left( \frac{10x^2}{x-y} \right) \bigg|_{(4,-1)} \\ 
                &= \frac{10x^2}{(x-y)^2} \bigg|_{(4,-1)} \\
                &= \frac{10 \times 4^2}{(4 - (-1))^2} = \frac{160}{25} = 6.4
    \end{align*}
    Hence, the linear approximation is given by \[ 
        \boxed{ L(x,y) = 32 + 9.6(x-4) + 6.4(y+1) }
    \]
}


%%%%%%%%%%%%%%%%%%%%%%%%%%%%%%%%%%%%%%%%%%%%%%%%%%%%%%%
%  Gradient Vector, Tangent Planes, and Normal Lines  %
%%%%%%%%%%%%%%%%%%%%%%%%%%%%%%%%%%%%%%%%%%%%%%%%%%%%%%%

\subsection{Gradient Vector, Tangent Planes, and Normal Lines}

\Note{
    The gradient vector $\nabla f(x_0,y_0)$ is orthogonal to the level curve $f(x,y)=k$ at the point $(x_0,y_0)$. Likewise, the gradient vector $\nabla f(x_0,y_0,z_0)$ is orthogonal to the level surface $f(x,y,z)=k$ at the point $(x_0,y_0,z_0)$.
}

We know, the gradient vector is \[ 
    \nabla f = \langle f_x, f_y, f_z \rangle
\]
So, the tangent plane to the surface $f(x,y,z)=k$ at $(x_0,y_0,z_0)$ is given by the equation \[ 
    f_x(x_0,y_0,z_0)(x-x_0) + f_y(x_0,y_0,z_0)(y-y_0) + f_z(x_0,y_0,z_0)(z-z_0) = 0
\]
This is a general form of the equation of a tangent plane than that of the previous section.

However, we can also write the equation in the previous form. For that, we need to find the tangent plane to the surface given by $z=f(x,y)$ at the point $(x_0,y_0,z_0)$, where $z_0=f(x_0,y_0)$. We can rewrite as \[ 
    f(x,y) - z = 0
\]
Now, defining a new function as \[ 
    F(x,y,z) = f(x,y) - z
\]

The surface given by $z=f(x,y)$ is identical to the surface given by $F(x,y,z)=0$. Now, the gradient vector for $F$ is \[ 
    \nabla F = \langle F_x, F_y, F_z \rangle = \langle f_x, f_y, -1 \rangle
\]
Note:
\begin{align*}
    F_x &= \pdv{}{x} \left[ f(x,y) - z \right] = f_x \\
    F_y &= \pdv{}{y} \left[ f(x,y) - z \right] = f_y \\
    F_z &= \pdv{}{z} \left[ f(x,y) - z \right] = -1
\end{align*}

Hence, the tangent plane is then
    \[ F_x (x-x_0) + F_y (y-y_0) + F_z (z-z_0) = 0 \]
    \[ f_x(x_0,y_0)(x-x_0) + f_y(x_0,y_0)(y-y_0) - (z-z_0) = 0 \]
    \[ \boxed{ z = f(x_0,y_0) + f_x(x_0,y_0)(x-x_0) + f_y(x_0,y_0)(y-y_0) } \]

We can also find the normal line to the surface $f(x,y,z)=k$ at the point $(x_0,y_0,z_0)$. The equation of line requires a point and a parallel vector, which is given by the gradient vector. Hence, the equation of the normal line is given by
\[ 
    \boxed{ \vec{r}(t) = \langle x_0,y_0,z_0 \rangle + t \nabla f(x_0,y_0,z_0) }
\]

\Note[Tangent Plane]{
    The tangent plane to the surface given by $F(x,y,z)=f(x,y)-z$ at the point $(x_0,y_0,z_0)$ is given by
    \[ 
        F_x(x_0,y_0,z_0)(x-x_0) + F_y(x_0,y_0,z_0)(y-y_0) + F_z(x_0,y_0,z_0)(z-z_0) = 0
    \]
}

\Note[Normal Line]{
    The normal line to the surface given by $F(x,y,z)=f(x,y)-z$ at the point $(x_0,y_0,z_0)$ is given by
    \[ 
        \vec{r}(t) = \langle x_0,y_0,z_0 \rangle + t \nabla F(x_0,y_0,z_0)
    \]
}

\Example{Find the tangent plane and normal line to $9yz - \sqrt{x^2-8z} = xy^2 - 26$ at $(3,1,-2)$}{
    Rearranging the equation, we get \[ 
        xy^2 + \sqrt{x^2-8z} - 9yz - 26 = 0
    \]
    The gradient vector is given by
    \begin{align*}
        \nabla F &= \langle y^2 + \frac{x}{\sqrt{x^2-8z}}, 2xy - 9z, - \frac{4}{\sqrt{x^2-8z}} - 9y \rangle \\
        \nabla F(3,1,-2) &= \langle 1 + \frac{3}{5}, 6+18, - \frac{4}{5} - 9 \rangle \\ 
                         &= \langle \frac{8}{5}, 24, \frac{-49}{5} \rangle
    \end{align*}
    The tangent plane, hence, would be 
    \begin{align*}
        \frac{8}{5}(x-3) + 24(y-1) - \frac{49}{5}(z+2) &= 0 \\
        8x - 24 + 120y - 120 - 49z - 98 &= 0 \\ 
        8x + 120y - 49z &= 242
    \end{align*}
    And the normal line is given by the equation 
    \[ \frac{x-3}{ \frac{8}{5} } = \frac{y-1}{24} = \frac{z+2}{ - \frac{49}{g} } \]
    \[ \frac{x-3}{8} = \frac{y-1}{120} = -\frac{z+2}{49} \]
}

\Example{Find the point(s) on $6x^2+y^2-3z^2=4$ where the tangent plane to the surface is parallel to the plane given by $2x+7y-z=6$.}{
    The gradient vector of the surface is given by \[ 
        \nabla F = \langle 12x, 2y, -6z \rangle
    \]
    The normal vector to the parallel plane is \[ 
        \vec{n} = \langle 2,7,-1 \rangle
    \]
    Since the planes are parallel, the gradient vector must be a scalar multiple of the normal vector, i.e., \[ 
        \langle 12x, 2y, -6z \rangle = k \langle 2,7,-1 \rangle
    \]
    This gives us the following equations:
    \[ 12x = 2k \implies k = 6x \]
    \[ 2y = 7k = 42x \implies y = 21x \]
    \[ -6z = -k \implies z = x \]
    Now, substituting these values in the equation of the surface, we get 
    \begin{align*}
        6x^2 + (21x)^2 - 3x^2 &= 4 \\
        6x^2 + 441x^2 - 3x^2 &= 4 \\
        x^2 &= \frac{4}{444}
    \end{align*}
    \[ \therefore x = \pm \frac{1}{\sqrt{111}}, \qquad y = \pm \frac{21}{\sqrt{111}}, \qquad z = \pm \frac{1}{\sqrt{111}} \]
    Hence, the points are \[ 
        \left( \frac{1}{\sqrt{111}}, \frac{21}{\sqrt{111}}, \frac{1}{\sqrt{111}} \right) \text{ and } \left( -\frac{1}{\sqrt{111}}, - \frac{21}{\sqrt{111}}, -\frac{1}{\sqrt{111}} \right)
    \]
}


%%%%%%%%%%%%%%%%%%%%%%%%%%%%%%%%%%%%
%  Relative Minimums and Maximums  %
%%%%%%%%%%%%%%%%%%%%%%%%%%%%%%%%%%%%

\subsection{Relative Minimums and Maximums}

\Definition{Relative Minimum and Maximum}{
    A function $f(x,y)$ has a \textbf{relative minimum} at the point $(a,b)$ if $f(x,y) \geq f(a,b)$ for all points $x(,y)$ in some region around $(a,b)$. \\~\\
    A function $f(x,y)$ has a \textbf{relative maximum} at the point $(a,b)$ if $f(x,y) \leq f(a,b)$ for all points $x(,y)$ in some region around $(a,b)$.
}

\Definition{Critical Point}{
    The point $(a,b)$ is a \textbf{critical point} (or a \textbf{stationary point}) of the function $f(x,y)$ provided one of the following is true:
    \begin{enumerate}
        \item $\nabla f(a,b) = \vec{0}$ (i.e., $f_x(a,b)=0$ and $f_y(a,b)=0$)
        \item $f_x(a,b)$ and/or $f_y(a,b)$ do not exist
    \end{enumerate}
}

\Note{
    If the point $(a,b)$ is a relative extrema of the function $f(x,y)$ and the first order derivative of $f(x,y)$ exists at $(a,b)$, then $(a,b)$ is a critical point of $f(x,y)$. However, the converse is not true. That is, a critical point need not be a relative extrema.
}

\underline{\textbf{Proof:}} \\ 
Let $g(x) = f(x,y)$ and suppose that $f(x,y)$ has a relative extrema at $(a,b)$. However, this also means that $g(x)$ also has a relative extrema at $x=a$. By Fermat's theorem, we know that $g'(a)=0$. But we also know that $g'(a)=f_x(a,b)$ and so we have that $f_x(a,b)=0$. 

If we now define $h(y)=f(a,b)$, then we can similarly show that $f_y(a,b)=0$.

So, putting all these together means that $\nabla f(a,b)=\vec{0}$ and so $f(x,y)$ has a critical point at $(a,b)$.

\begin{minipage}{0.6\textwidth}
    Example of a point that is a critical point but not a relative extrema is the point $(0,0)$ for the function $f(x,y)=x^2-y^2$. In the $x$-axis, the function becomes $f(x,0)=x^2$, which is a parabola opening upwards. So, the point $(0,0)$ looks like a minimum along this path. In the $y$-axis, the function becomes $f(0,y)=-y^2$, which is a parabola opening downwards. So, the point $(0,0)$ looks like a maximum along this path. Since the point $(0,0)$ is a minimum in one direction and a maximum in another, it's neither a relative minimum nor a relative maximum. This type of critical point is also called a \textbf{saddle point}.
\end{minipage}
\begin{minipage}{0.4\textwidth}
    \begin{center}
        \includegraphics[width=0.8\textwidth]{3.3.1.png}
    \end{center}
    \captionof{figure}{Saddle point}
\end{minipage}

\Definition{Saddle Point}{
    A point $(a,b)$ is a \textbf{saddle point} of the function $f(x,y)$ if it is a critical point but neither a relative minimum nor a relative maximum.
}

\Definition{Hessian Matrix}{
    The Hessian matrix of a function $f(x,y)$ is given by \[ 
        H = \begin{bmatrix}
            f_{xx} & f_{xy} \\
            f_{xy} & f_{yy}
        \end{bmatrix}
    \]
    where $f_{xx}$, $f_{yy}$, and $f_{xy}$ are the second order partial derivatives of $f(x,y)$. \\
    The Hessian matrix is used in the second derivative test to classify critical points of the function $f(x,y)$. 
}

\Note[Relative Extrema]{
    Suppose that $(a,b)$ is a critical point of $f(x,y)$ and that the second order partial derivatives are continuous in some region that contains $(a,b)$. Next, let's use the determinant of the Hessian matrix, denoted by $D$, to classify the critical point $(a,b)$.
    \[ 
        D = D(a,b) = f_{xx}(a,b)f_{yy}(a,b) - \left[ f_{xy}(a,b) \right]^2
    \]
    We then have the following classifications of the critical point:
    \begin{enumerate}
        \item If $D>0$ and $f_{xx}(a,b)>0$ then there is a relative minimum at $(a,b)$. 
        \item If $D>0$ and $f_{xx}(a,b)<0$ then there is a relative maximum at $(a,b)$. 
        \item If $D<0$ then there is a saddle point at $(a,b)$. 
        \item If $D=0$ then the test is inconclusive. In this case, we need to use other methods to determine the nature of the critical point.
    \end{enumerate}
}

Note that if $D>0$ then both $f_{xx}(a,b)$ and $f_{yy}(a,b)$ will have the same sign and so in the first two cases, we can conclude that both $f_{xx}(a,b)$ and $f_{yy}(a,b)$ are either both positive or both negative.

\Example{Find and classify all the critical points for $f(x,y) = 3x^2y + y^3 - 3x^2 - 3y^2 + 2$}{
    First, we need to find the first and second order partial derivatives of the function: \[ 
        f_x = 6xy - 6x, \qquad f_y = 3x^2 + 3y^2 - 6y
    \] \[ 
        f_{xx} = 6y - 6, \qquad f_{yy} = 6y - 6, \qquad f_{xy} = 6x
    \]
    Now, we need to find the critical points by solving the equations $f_x=0$ and $f_y=0$ simultaneously:
    \begin{align*}
        6xy - 6x &= 0 \\
        3x^2 + 3y^2 - 6y = 0
    \end{align*}
    From the first equation, \[ 
        6x(y-1) = 0 \implies x=0, y=1
    \]
    For $x=0$:
    \[ 3y^2 - 6y = 3y(y-2) = 0 \implies y = 0,2 \]
    For $y=1$:
    \[ 3x^2 + 3 - 6 = 0 \implies x^2 = 1 \implies x = 1,-1 \]
    Hence, the critical points are $(0,0)$, $(0,2)$, $(1,1)$, and $(-1,1)$.
    Now, we need to classify these critical points using the second derivative test: \[ 
        D(x,y) = f_{xx}(x,y)f_{yy}(x,y) - \left[ f_{xy}(x,y) \right]^2 = (6y-6)^2 - 36x^2
    \]
    For $(0,0)$:
    \[ D(0,0) = 36 > 0 \qquad f_{xx}(0,0) = -6 < 0 \]
    For $(0,2)$:
    \[ D(0,2) = 36 > 0 \qquad f_{xx}(0,2) = 6 > 0 \]
    For $(1,1)$:
    \[ D(1,1) = -36 < 0 \]
    For $(-1,1)$:
    \[ D(-1,1) = -36 < 0 \]
    So, we have the following classifications:
    \begin{itemize}
        \item $(0,0)$ is a relative maximum.
        \item $(0,2)$ is a relative minimum.
        \item $(1,1)$ and $(-1,1)$ are saddle points. 
    \end{itemize}
}

\Example{Determine the point on the plane $4x-2y+z=1$ that is closest to the point $(-2,-1,5)$}{
    We can rewrite the equation of the plane as \[ 
        z = 1 - 4x + 2y
    \]
    The distance between a point $(x,y,z)$ on the plane and the point $(-2,-1,5)$ is given by \[ 
        d = \sqrt{(x+2)^2 + (y+1)^2 + (2y-4x-4)^2}
    \]
    Now, since the finding the mininmum value of $d$ is equivalent to finding the minimum value of $d^2$, let
    \begin{align*}
        f(x,y) &= d^2 = (x+2)^2 + (y+1)^2 + (2y-4x-4)^2 \\
        &= (x^2 + 4x + 4) + (y^2 + 2y + 1) + (16x^2 + 4y^2 - 16xy + 32x - 16y + 16) \\
        f(x,y) &= 17x^2 + 5y^2 - 16xy + 36x - 14y + 21
    \end{align*}
    We can now find the derivatives:
    \[ f_x(x,y) = 34x-16y+36, \qquad f_y(x,y) = 10y-16x-14 \]
    \[ f_{xx}(x,y) = 34, \qquad f_{yy}(x,y) = 10, \qquad f_{xy} = -16 \]
    Before finding the critical points, notice that
    \begin{align*}
        D(x,y) &= f_{xx}f_{yy} - f_{xy}^2 \\
        &= 34 \times 10 - (-16)^2 \\
        &= 340 - 256 = 84 > 0
    \end{align*}
    Since $D>0$ and $f_{xx}>0$, all the critical points will be relative minimums. Now, to find the critical points, solve the equations
    \begin{align*}
        f_x &= 34x - 16y + 36 = 0 \\
        f_y &= 10y - 16x - 14 = 0
    \end{align*}
    From the first equation, we get \[ 
        x = \frac{1}{34}(16y - 36) = \frac{1}{17}(8y-18)
    \]
    Substituting this in the second equation, we get
    \begin{align*}
        10y - \frac{16}{17}(8y-18) - 14 &= 0 \\
        170y - 16(8y-18) - 238 &= 0 \\ 
        42y + 50 &= 0 \\ 
        \therefore y &= -\frac{25}{21}
    \end{align*}
    And substituting this value in the equation for $x$, we get
    \begin{align*}
        x &= \frac{1}{17} \left( \frac{-200}{21} - 18 \right) \\
        &= \frac{-200-378}{17 \times 21} \\
        \therefore x &= -\frac{34}{21}
    \end{align*}
    Finally, substituting these values in the equation of the plane, we get \[ 
        z = 1 + 4 \times \frac{34}{21} - 2 \times \frac{25}{21} = \frac{107}{21}
    \]
    So, the point on the plane that is closest to the point $(-2,-1,5)$ is $\displaystyle \left( -\frac{34}{21}, -\frac{25}{21}, \frac{107}{21} \right)$
}

%%%%%%%%%%%%%%%%%%%%%%
%  Absolute Extrema  %
%%%%%%%%%%%%%%%%%%%%%%

\subsection{Absolute Extrema}

\Definition{Closed, Open, and Bounded Region}{
    \begin{enumerate}
        \item A region $\R^2$ is called \textbf{closed} if it contains all its boundary points. 
        \item A region $\R^2$ is called \textbf{open} if it does not contain any of its boundary points. 
        \item A region $\R^2$ is called \textbf{bounded} if it can be contained in a circle of finite radius.
    \end{enumerate}
}

\Theorem{Extreme Value Theorem}{
    If $f(x,y)$ is continuous on a closed and bounded region $D$ in $\R^2$, then $f(x,y)$ there are two points $(x_1,y_1)$ and $(x_2,y_2)$ in region $D$ such that $f(x_1,y_1)$ is the absolute maximum and $f(x_2,y_2)$ is the absolute minimum of $f(x,y)$ in $D$.
}

\Note[Finding Absolute Extrema]{
    \begin{enumerate}
        \item Find all the critical points of the function $f(x,y)$ that lie in the region $D$ and determine the function value at each of these points. 
        \item Find all extrema of the function $f(x,y)$ on the boundary of the region $D$ and determine the function value at each of these points. 
        \item The largest and the smallest values found in the first two steps are the absolute minimum and the absolute maximum of the function $f(x,y)$ in the region $D$.
    \end{enumerate}
}

\Example{Find the absolute minimum and absolute maximum of $f(x,y)=x^2+4y^2-2x^2y+4$ on the rectangle given by $-1 \leq x \leq 1$ and $-1 \leq y \leq 1$.}{
    First, we need to find the critical points of the function $f(x,y)$ in the region $D$. The first order partial derivatives are given by
    \begin{align*}
        f_x &= 2x - 4xy \\
        f_y &= 8y - 2x^2
    \end{align*}
    Setting these equal to zero, we get the following equations:
    \begin{align*}
        2x - 4xy &= 0 \\
        8y - 2x^2 &= 0
    \end{align*}
    Solving the equations, we get \[ 
        (x,y) = (0,0), \left(\sqrt{2}, \frac{1}{2}\right), \left(-\sqrt{2}, \frac{1}{2}\right)
    \] Since $-1 \leq x \leq 1$, the only critical point in the region $D$ is $(0,0)$.
    Value of the function at this point is \[ 
        f(0,0) = 0^2 + 4(0)^2 - 2(0)^2(0) + 4 = 4
    \]
    Now, we need to find the extrema on the boundary of the region. The boundary consists of four line segments: \\
    \begin{minipage}{0.6\textwidth}
        \begin{enumerate}
            \item $(-1,-1)$ to $(1,-1)$ where $y=-1$ and $-1\leq x\leq 1$
            \item $(1,-1)$ to $(1,1)$ where $x=1$ and $-1\leq y\leq 1$
            \item $(1,1)$ to $(-1,1)$ where $y=1$ and $-1\leq x\leq 1$
            \item $(-1,1)$ to $(-1,-1)$ where $x=-1$ and $-1\leq y\leq 1$
        \end{enumerate}
    \end{minipage}
    \begin{minipage}{0.3\textwidth}
        \begin{center}
            \includegraphics[width=0.8\textwidth]{3.4.1.png}
            \captionof{figure}{Boundary of the region}
        \end{center}
    \end{minipage} \\
    For segment 1: \\
    Let $g(x)=f(x,-1)=3x^2+8$ \\
    For critical point on the edge, $g'(x)=0$ or $6x=0 \implies x=0$. \\
    \begin{align*}
        g(0) &= f(0,-1) = 8 \\
        g(-1) &= f(-1,-1) = 11 \\ 
        g(1) &= f(1,-1) = 11
    \end{align*}
    For segment 2: \\
    Let $h(y)=f(1,y)=1+4y^2-2(1)y+4=5+4y^2-2y$ \\ 
    For critical point on the edge, $h'(y)=0$ or $8y-2=0 \implies y=\frac{1}{4}$. \\ 
    \begin{align*}
        h\left(\frac{1}{4}\right) &= f(1,\frac{1}{4}) = \frac{19}{4} = 4.75 \\
        h(-1) &= f(1,-1) = 11 \\ 
        h(1) &= f(1,1) = 7
    \end{align*}
    For segment 3: \\ 
    Let $k(x)=f(x,1)=x^2+4(1)^2-2x^2(1)+4=5-2x^2$ \\ 
    For critical point on the edge, $k'(x)=0$ or $-4x=0 \implies x=0$. \\ 
    \begin{align*}
        k(0) &= f(0,1) = 8 \\
        k(-1) &= f(-1,1) = 7 \\ 
        k(1) &= f(1,1) = 7
    \end{align*}
    For segment 4: \\ 
    Let $l(y)=f(-1,y)=(-1)^2+4y^2-2(-1)y+4=5+4y^2+2y$ \\ 
    For critical point on the edge, $l'(y)=0$ or $8y+2=0 \implies y=-\frac{1}{4}$. \\ 
    \begin{align*}
        l\left(-\frac{1}{4}\right) &= f(-1,-\frac{1}{4}) = \frac{19}{4} = 4.75 \\
        l(-1) &= f(-1,-1) = 11 \\ 
        l(1) &= f(-1,1) = 7
    \end{align*}
    Now, we can summarize the function values at the critical point and the boundary points:
    \begin{align*}
        & f(0,0) = 4, \\ 
        & f(0,-1) = 8,      && f(-1,-1) = 11,       &&& f(1,-1) = 11, \\ 
        & f\left(1,\frac{1}{4}\right) = 4.75,       && f(1,-1) = 11,      &&& f(1,1) = 7, \\ 
        & f(0,1) = 8,       && f(-1,1) = 7,         &&& f(1,1) = 7, \\ 
        & f\left(-1,-\frac{1}{4}\right) = 4.75,     && f(-1,-1) = 11,       &&& f(-1,1) = 7
    \end{align*}
    Hence, the absolute extrema are given by
    \begin{align*}
        &\text{Absolute Maximum: } (-1,-1) \text{ and } (1,-1) \text{ with value } 11, \\ 
        &\text{Absolute Minimum: } (0,0) \text{ with value } 4 
    \end{align*}
}

\Example{Find the absolute minimum and absolute maximum of $f(x,y)=2x^2-y^2+6y$ on the disk of radius $4$, $x^2+y^2 \leq 16$}{
    First, we need to find the critical points of the function $f(x,y)$ in the region $D$. The first order partial derivatives are given by
    \begin{align*}
        f_x &= 4x \\
        f_y &= -2y + 6
    \end{align*}
    Setting these equal to zero, we get the following equations:
    \begin{align*}
        4x &= 0 \implies x = 0 \\
        -2y + 6 &= 0 \implies y = 3
    \end{align*}
    So, the only critical point in the region $D$ is $(0,3)$. The value of the function at this point is \[ 
        f(0,3) = 2(0)^2 - (3)^2 + 6(3) = 15
    \]
    Now, we need to find the extrema on the boundary of the region. The boundary consists of the circle $x^2+y^2=16$. We can rewrite it as \[ 
        x^2 = 16 - y^2
    \]
    Let
    \begin{align*}
        g(y) &= f\left(\sqrt{16-y^2},y\right) \\ 
             &= 2(16-y^2) - y^2 + 6y \\ 
             &= 32 - 3y^2 + 6y 
    \end{align*}
    The critical points on the boundary are given by \[ 
        g'(y) = -6y + 6 = 0 \implies y = 1
    \]
    \[ x = \pm \sqrt{y^2-1} = \pm \sqrt{15} \]
    Now, we can find the value of the function at this point and at the endpoints of the boundary:
    \begin{align*}
        g(1) &= f(\pm \sqrt{15},1) = 35 \\
        g(-4) &= f(0,-4) = -40 \\ 
        g(4) &= f(0,4) = 8
    \end{align*}
    Now, we can summarize the function values at the critical point and the boundary points:
    \begin{align*}
        & f(0,3) = 15, \\ 
        & f(\sqrt{15},1) = 35, && f(-\sqrt{15},1) = 35, \\ 
        & f(0,-4) = -40, && f(0,4) = 8
    \end{align*}
    Hence, the absolute extrema are given by
    \begin{align*}
        &\text{Absolute Maximum: } (\sqrt{15},1) \text{ and } (-\sqrt{15},1) \text{ with value } 35, \\ 
        &\text{Absolute Minimum: } (0,-4) \text{ with value } -40
    \end{align*}
}

\Example{Find the absolute minimum and absolute maximum of $f(x,y)=18x^2+4y^2-y^2x-2$ on the triangle with vertices $(-1,-1)$, $(5,-1)$, and $(5,17)$.}{
    First, we need to find the critical points of the function $f(x,y)$ in the region $D$. The first order partial derivatives are given by
    \begin{align*}
        f_x &= 3x - y^2 \\
        f_y &= 8y - 2xy
    \end{align*}
    Setting these equal to zero, we get the following equations: 
    \begin{align*}
        3x - y^2 &= 0 \\
        8y - 2xy &= 0 
    \end{align*}
    Solving the equations, we get \[ 
        (x,y) = (0,0), (4,-12), (4,12)
    \]
    Notice that the triangle is bounded by \\
    \begin{minipage}{0.63\textwidth}
        \begin{enumerate}
            \item $y = -1$
            \item $x = 5$
            \item $y+1 = 3(x+1) \implies y = 3x+2$
        \end{enumerate}
        Among the critical points,
        \begin{enumerate}
            \item $(0,0)$ lies inside the triangle. 
            \item $(4,12)$ lies inside the triangle. $[3 \times 4 + 2 = 14 > 12]$ 
            \item $(4,-12)$ lies outside the triangle. $[-1 \leq y \leq 17]$
        \end{enumerate}
    \end{minipage}
    \begin{minipage}{0.34\textwidth}
        \begin{center}
            \includegraphics[width=0.5\textwidth]{3.4.2.png}
            \captionof{figure}{Triangle region}
        \end{center}
    \end{minipage} \\
    The values of the function at the critical points are: \[ 
        f(0,0) = -2 \qquad \text{ and } \qquad f(4,12) = 286
    \]
    Now, we need to find the extrema on the boundary of the region. The boundary consists of three line segments: \\
    \begin{enumerate}
        \item $(-1,-1)$ to $(5,-1)$ where $y=-1$ and $-1\leq x\leq 5$ 
        \item $(5,-1)$ to $(5,17)$ where $x=5$ and $-1\leq y\leq 17$ 
        \item $(-1,-1)$ to $(5,17)$ where $y=3x+2$ and $-1\leq x\leq 5$
    \end{enumerate}
    For segment 1: \\ 
    Let $g(x)=f(x,-1) = 18x^2 + 4 - x - 2 = 18x^2 - x + 2$ \\
    For critical point on the edge, $g'(x)=0$ or $36x-1=0 \implies x=\frac{1}{36}$. \\
    \begin{align*}
        g\left(\frac{1}{36}\right) &= f\left(\frac{1}{36},-1\right) \approx 1.986 \\
        g(-1) &= f(-1,-1) = 21 \\ 
        g(5) &= f(5,-1) = 447
    \end{align*}
    For segment 2: \\ 
    Let $h(y)=f(5,y) = 450 + 4y^2 - 5y^2 - 2 = 448 - 9y^2$ \\ 
    For critical point on the edge, $h'(y) = 0$ or $\displaystyle 448 - 9y^2 = 0 \implies y = \pm \frac{8\sqrt{7}}{3}$
    \begin{align*}
        h\left(\frac{8\sqrt{7}}{3}\right) &= f\left(5,\frac{8\sqrt{7}}{3}\right) \approx 398.22 \\
        h\left(-\frac{8\sqrt{7}}{3}\right) &= f\left(5,-\frac{8\sqrt{7}}{3}\right) \approx 398.22 \\
        h(-1) &= f(5,-1) = 447 \\ 
        h(17) &= f(5,17) = 159
    \end{align*}
    For segment 3: \\ 
    Let $k(x)=f(x,3x+2) = 18x^2 + 4(3x+2)^2 - (3x+2)^2x - 2$ \\
    \begin{align*}
        k(x) &= 18x^2 + 4(3x+2)^2 - (3x+2)^2x - 2 \\
             &= 18x^2 + 4(9x^2 + 12x + 4) - (9x^3 + 12x^2 + 4x) - 2 \\
             &= 18x^2 + 36x^2 + 48x + 16 - 9x^3 - 12x^2 - 4x - 2 \\
             &= -9x^3 + 42x^2 + 44x + 14
    \end{align*}
    For critical point on the edge, $k'(x)=0$ or $-27x^2 + 84x + 44 = 0$. This gives us the points $(x,y) \approx (3.568,12.704), (-0.457,0.629)$
    \begin{align*}
        k(3.568) &= f(3.568,12.704) \approx 296.872 \\
        k(-0.457) &= f(-0.457,0.629) \approx 3.523
    \end{align*}
    Hence, the absolute extrema are given by
    \begin{align*}
        &\text{Absolute Maximum: } (5,\pm1) \text{ with value } 447, \\ 
        &\text{Absolute Minimum: } (0,0) \text{ with value } -2
    \end{align*}
}

%%%%%%%%%%%%%%%%%%%%%%%%%%
%  Lagrange Multipliers  %
%%%%%%%%%%%%%%%%%%%%%%%%%%

\subsection{Lagrange Multipliers}

\Note[Method of Lagrange Multipliers]{
    Given a function $f(x,y,z)$ and a constraint $g(x,y,z)=k$. To find the absolute extrema:
    \begin{enumerate}
        \item Solve the following system of equations:
            \begin{align*}
                \nabla f(x,y,z) &= \lambda \nabla g(x,y,z) \\
                g(x,y,z) &= k
            \end{align*} 
        \item Plug in all solutions, $(x,y,z)$, from the first step into the function $f(x,y,z)$ and identify the minimum and maximum values, provided they exist and $\nabla g \neq \vec{0}$ at the point.
    \end{enumerate}
    The constant $\lambda$ is called the \textbf{Lagrange Multiplier}.
}

\Note[Lagrange Multipliers for Multiple Constraints]{
    If there are multiple constraints, say $g_1(x,y,z)=k_1$ and $g_2(x,y,z)=k_2$, then the system of equations becomes
    \begin{align*}
        \nabla f(x,y,z) &= \lambda_1 \nabla g_1(x,y,z) + \lambda_2 \nabla g_2(x,y,z) \\
        g_1(x,y,z) &= k_1 \\
        g_2(x,y,z) &= k_2
    \end{align*}
    where $\lambda_1$ and $\lambda_2$ are the Lagrange Multipliers for the constraints $g_1$ and $g_2$, respectively. \\
    The method can be extended to any number of constraints by adding more terms to the right-hand side of the first equation. 
}

\Example{Find the maximum and minimum values of $f(x,y) = 4x^2+10y^2$ on the disk $x^2+y^2 \leq 4$.}{
    Using Lagrange multipliers, we need to solve the following system of equations:
    \begin{align*}
        \nabla f(x,y) &= \lambda \nabla g(x,y) \\
        g(x,y) &= x^2+y^2 = 4
    \end{align*}
    That is, we need to solve the following equations:
    \begin{align*}
        8x &= 2\lambda x \\ 
        20y &= 2\lambda y \\ 
        x^2 + y^2 &= 4
    \end{align*}
    From the first equation, we get: \[ 
        2x(4-\lambda) = 0 \implies x=0 \text{ or } \lambda = 4
    \]
    If we have $x=0$ then the constraint gives us \[ 
        y = \sqrt{x^2 - 4} \implies y = \pm 2
    \]
    If we have $\lambda=4$ the second equation gives us \[ 
        20y = 8y \implies y = 0
    \]
    Then the constraint gives us \[ 
        x^2 + 0^2 = 4 \implies x = \pm 2
    \]
    Now, we can find the values of the function at these points:
    \begin{align*}
        & f(0,0) = 0 \\
        & f(0,2) = 40    && f(0,-2) = 40 \\
        & f(2,0) = 16    && f(-2,0) = 16
    \end{align*}
    Hence, the maximum value is $40$ at the points $(0,2)$ and $(0,-2)$, and the minimum value is $0$ at the point $(0,0)$.
}

\Example{Find the maximum and minimum values of $f(x,y,z) = 4y-2z$ subject to the constraints $2x-y-z = 2$ and $x^2+y^2 = 1$.}{
    Using Lagrange multipliers method, we need to solve the following system of equations:
    \begin{align*}
        \nabla f(x,y,z) &= \lambda \nabla g(x,y,z) + \mu \nabla h(x,y,z)  \\
        g(x,y,z) &= 2x - y - z = 2 \\ 
        h(x,y,z) &= x^2 + y^2 = 1
    \end{align*}
    That is, we need to solve the following equations:
    \begin{align*}
        0 &= 2\lambda + 2\mu x \\
        4 &= -\lambda + 2\mu y \\ 
        -2 &= -\lambda \\ 
        2x - y - z &= 2 \\ 
        x^2 + y^2 &= 1
    \end{align*}
    From the third equation, we get $\lambda=2$. Plugging this into the first and second equations, we get respectively
    \begin{align*}
        -4 &= 2\mu x \implies x = - \frac{2}{\mu} \\
        6 &= 2\mu y \implies y = \frac{3}{\mu}
    \end{align*}
    Now, substituting these values in the fifth equation, we get \[ 
        \frac{4}{\mu^2} + \frac{9}{\mu^2} = 1 \implies \mu = \pm \sqrt{13}
    \]
    Now, for $\mu = \sqrt{13}$, we have
    \begin{align*}
        x &= -\frac{2}{\sqrt{13}} \\
        y &=  \frac{3}{\sqrt{13}} \\
        z &= 2x - y - 2 = -\frac{4}{\sqrt{13}} - \frac{3}{\sqrt{13}} - 2 = - 2 - \frac{7}{\sqrt{13}}
    \end{align*}
    And for $\mu = -\sqrt{13}$, we have
    \begin{align*}
        x &= \frac{2}{\sqrt{13}} \\
        y &= -\frac{3}{\sqrt{13}} \\ 
        z &= 2x - y - 2 = \frac{4}{\sqrt{13}} + \frac{3}{\sqrt{13}} - 2 = -2 + \frac{7}{\sqrt{13}}
    \end{align*}
    Now, we can find the values of the function at these points:
    \begin{align*}
        f \left(-\frac{2}{\sqrt{13}}, \frac{3}{\sqrt{13}}, -2 - \frac{7}{\sqrt{13}}\right) &\approx 11.211 \\
        f \left(\frac{2}{\sqrt{13}}, -\frac{3}{\sqrt{13}}, -2 + \frac{7}{\sqrt{13}}\right) &\approx -3.211
    \end{align*}
    Hence, the maximum value is approximately $11.211$ at the point $\left(-\frac{2}{\sqrt{13}}, \frac{3}{\sqrt{13}}, -2 - \frac{7}{\sqrt{13}}\right)$, and the minimum value is approximately $-3.211$ at the point $\left(\frac{2}{\sqrt{13}}, -\frac{3}{\sqrt{13}}, -2 + \frac{7}{\sqrt{13}}\right)$.
}
