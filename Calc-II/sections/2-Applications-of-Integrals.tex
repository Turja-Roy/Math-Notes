\section{Applications of Integrals}

%%%%%%%%%%%%%%%%
%  Arc Length  %
%%%%%%%%%%%%%%%%

\subsection{Arc Length}

Consider a curve $y=f(x)$. We want to find the length of the curve from $x=a$ to $x=b$. We can approximate the curve by a series of line segments. The length of each line segment is given by the Pythagorean theorem: \[ 
    \dd{s} = \sqrt{\dd{x}^2 + \dd{y}^2} = \sqrt{1 + \left( \dv{y}{x} \right)^2} \dd{x}
\]
The total length of the curve is given by the sum of the lengths of the line segments: \[ 
    L = \int \dd{s} = \int \sqrt{1 + \left( \dv{y}{x} \right)^2} \dd{x}
\]
Similarly, for a curve $x=h(y)$, the length of the curve from $y=c$ to $y=d$ is given by: \[ 
    L = \int \dd{s} = \int \sqrt{1 + \left( \dv{x}{y} \right)^2} \dd{y}
\]
\Note[Arc Length Formula]{
    \[ 
        L = \int \dd{s}
    \] where, 
    \begin{align*}
        ds &= \sqrt{1 + \left( \dv{y}{x} \right)^2} \dd{x} \quad \text{for } y = f(x), a \leq x \leq b \\
        ds &= \sqrt{1 + \left( \dv{x}{y} \right)^2} \dd{y} \quad \text{for } x = h(y), c \leq y \leq d
    \end{align*}
}


%%%%%%%%%%%%%%%%%%
%  Surface Area  %
%%%%%%%%%%%%%%%%%%

\subsection{Surface Area}

Consider a curve $y=f(x)$ rotated about the $x$-axis. We want to find the surface area of the resulting surface. We can approximate the surface by a series of frustums. The surface area of each frustum is given by: \[ 
    \dd{S} = 2\pi y \dd{s} = 2\pi f(x) \sqrt{1 + \left( \dv{y}{x} \right)^2} \dd{x}
\]
The total surface area of the surface is given by the sum of the surface areas of the frustums: \[ 
    S = \int \dd{S} = \int 2\pi y \dd{s} = \int 2\pi f(x) \sqrt{1 + \left( \dv{y}{x} \right)^2} \dd{x}
\]
Similarly, for a curve $x=h(y)$ rotated about the $y$-axis, the surface area of the resulting surface is given by: \[ 
    A = \int 2\pi x \dd{s} = \int 2\pi h(y) \sqrt{1 + \left( \dv{x}{y} \right)^2} \dd{y}
\]

\Note[Surface Area Formula]{
    \begin{align*}
        S &= \int \dd{S} \\
          &= \int 2 \pi y \dd{s} \quad \text{rotation about } x \text{--axis} \\
          &= \int 2 \pi x \dd{s} \quad \text{rotation about } y \text{--axis}
    \end{align*}
    where, 
    \begin{align*}
        dS &= 2\pi y \dd{s} = 2\pi f(x) \sqrt{1 + \left( \dv{y}{x} \right)^2} \dd{x} \quad \text{for } y = f(x), a \leq x \leq b \\
        dS &= 2\pi x \dd{s} = 2\pi h(y) \sqrt{1 + \left( \dv{x}{y} \right)^2} \dd{y} \quad \text{for } x = h(y), c \leq y \leq d
    \end{align*}
}


%%%%%%%%%%%%%%%%%%%%
%  Center of Mass  %
%%%%%%%%%%%%%%%%%%%%

\subsection{Center of Mass}

Suppose we want to find the center of mass of a region bounded by two curves $f(x)$ and $g(x)$ on the interval $[a,b]$. \\ 
The mass is   \[ 
      M = \rho \int_{a}^{b} \left( f(x) - g(x) \right) \dd{x}
\]
Next, we need the \textbf{moments} of the region. There are two moments:
\begin{align*}
    M_x &= \rho \int_{a}^{b} \frac{1}{2} \left[ f(x)^2 - g(x)^2 \right] \dd{x} \\
    M_y &= \rho \int_{a}^{b} x \left[ f(x) - g(x) \right] \dd{x}
\end{align*}
The coordinates of the center of mass, $(\bar{x}, \bar{y})$, are given by:
\Note[Center of Mass Formula]{
    \[
        \bar{x} = \frac{M_y}{M} = \frac{1}{A} \int_{a}^{b} x \left[ f(x) - g(x) \right] \dd{x}
    \]
    \[
        \bar{y} = \frac{M_x}{M} = \frac{1}{A} \int_{a}^{b} \frac{1}{2} \left[ f(x)^2 - g(x)^2 \right] \dd{x}
    \]
    where,   \[ 
          A = \int_{a}^{b} \left[ f(x) - g(x) \right] \dd{x}
    \]
}


%%%%%%%%%%%%%%%%%
%  Probability  %
%%%%%%%%%%%%%%%%%

\subsection{Probability}

Every continuous random variable $X$, has a probability density function $f(x)$. Probability density functions satisfy the following conditions:
\begin{enumerate}
    \item $f(x) \geq 0$ for all $x$. 
    \item $\displaystyle \int_{-\infty}^{\infty} f(x) \dd{x} = 1$
\end{enumerate}

Probability density finctions can be used to determine the probability that a continuous random variable lies between two values, say $a$ and $b$. This probability is denoted by $P(a \leq X \leq b)$. 
\Note{
    \[ P(a \leq X \leq b) = \int_{a}^{b} f(x) \dd{x} \]
}

Probability density functions can also be used to determine the mean of a continuous random variable. The mean is given by:
\Note[Mean of a Continuous Random Variable]{
    \[ \mu = \int_{-\infty}^{\infty} x f(x) \dd{x} \]
}
