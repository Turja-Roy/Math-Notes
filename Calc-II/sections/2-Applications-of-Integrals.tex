\section{Applications of Integrals}

%%%%%%%%%%%%%%%%%%%%%
%  Volume by Slicing  %
%%%%%%%%%%%%%%%%%%%%%

\subsection{Volume by Slicing}

The general idea of finding volume by slicing is to cut a solid into thin slices perpendicular to an axis, find the area of each cross-section, and integrate to get the total volume.

\subsubsection{General Cross-Sections}

\Note[Volume by Cross-Sections]{
    If a solid lies between $x=a$ and $x=b$, and the cross-sectional area perpendicular to the $x$-axis at position $x$ is $A(x)$, then the volume is:
    \[ 
        V = \int_{a}^{b} A(x) \dd{x}
    \]
    Similarly, if cross-sections are perpendicular to the $y$-axis:
    \[ 
        V = \int_{c}^{d} A(y) \dd{y}
    \]
}

\Example{
    Find the volume of a solid whose base is the region bounded by $y = x^2$ and $y = 4$ (from $x = -2$ to $x = 2$), with cross-sections perpendicular to the $x$-axis that are squares.
}{
    The side length of each square is the height of the region: $s = 4 - x^2$. The area of each square cross-section is $A(x) = s^2 = (4 - x^2)^2$. The volume is:
    \begin{align*}
        V &= \int_{-2}^{2} (4 - x^2)^2 \dd{x} = \int_{-2}^{2} (16 - 8x^2 + x^4) \dd{x} \\
          &= \left[ 16x - \frac{8x^3}{3} + \frac{x^5}{5} \right]_{-2}^{2} \\
          &= 2\left( 32 - \frac{64}{3} + \frac{32}{5} \right) = \frac{512}{15}
    \end{align*}
}


\subsubsection{Disk Method}

When a region is rotated about an axis, if the cross-sections are solid disks, we use the \textbf{disk method}. 

For a curve $y = f(x)$ rotated about the $x$-axis from $x = a$ to $x = b$:
\begin{itemize}
    \item Radius of disk: $r = f(x)$
    \item Area of disk: $A(x) = \pi r^2 = \pi [f(x)]^2$
\end{itemize}

\Note[Disk Method Formula]{
    Rotation about the $x$-axis:
    \[ 
        V = \int_{a}^{b} \pi [f(x)]^2 \dd{x}
    \]
    Rotation about the $y$-axis:
    \[ 
        V = \int_{c}^{d} \pi [g(y)]^2 \dd{y}
    \]
}

\subsubsection{Washer Method}

When the solid has a hollow region (like a donut), we use the \textbf{washer method}. The cross-section is a washer (disk with a hole).

For a region between $y = f(x)$ (outer) and $y = g(x)$ (inner) rotated about the $x$-axis:
\begin{itemize}
    \item Outer radius: $R = f(x)$
    \item Inner radius: $r = g(x)$
    \item Area of washer: $A(x) = \pi R^2 - \pi r^2 = \pi ([f(x)]^2 - [g(x)]^2)$
\end{itemize}

\Note[Washer Method Formula]{
    Rotation about the $x$-axis:
    \[ 
        V = \int_{a}^{b} \pi \left( [R(x)]^2 - [r(x)]^2 \right) \dd{x}
    \]
    Rotation about the $y$-axis:
    \[ 
        V = \int_{c}^{d} \pi \left( [R(y)]^2 - [r(y)]^2 \right) \dd{y}
    \]
    where $R$ is the outer radius and $r$ is the inner radius.
}

\Example{
    Find the volume of the solid obtained by rotating the region bounded by $y = x^2$ and $y = 2x$ about the $x$-axis.
}{
    First, find intersection points: $x^2 = 2x \Rightarrow x = 0, 2$. \\
    For $0 \leq x \leq 2$, we have $2x \geq x^2$ (the line is above the parabola).
    \begin{itemize}
        \item Outer radius: $R(x) = 2x$
        \item Inner radius: $r(x) = x^2$
    \end{itemize}
    \begin{align*}
        V &= \int_{0}^{2} \pi \left( (2x)^2 - (x^2)^2 \right) \dd{x} \\
          &= \pi \int_{0}^{2} (4x^2 - x^4) \dd{x} \\
          &= \pi \left[ \frac{4x^3}{3} - \frac{x^5}{5} \right]_{0}^{2} \\
          &= \pi \left( \frac{32}{3} - \frac{32}{5} \right) = \frac{64\pi}{15}
    \end{align*}
}


%%%%%%%%%%%%%%%%%%%%%%%%%%%%%%%
%  Volume by Cylindrical Shells  %
%%%%%%%%%%%%%%%%%%%%%%%%%%%%%%%

\subsection{Volume by Cylindrical Shells}

The \textbf{shell method} is an alternative to the disk/washer method. Instead of slicing perpendicular to the axis of rotation, we slice parallel to it, creating cylindrical shells.

Consider rotating a region about the $y$-axis. Each vertical strip at position $x$ with height $h(x)$ and thickness $\dd{x}$ creates a cylindrical shell:
\begin{itemize}
    \item Radius of shell: $r = x$
    \item Height of shell: $h = f(x)$
    \item Circumference: $2\pi r = 2\pi x$
    \item Volume of shell: $\dd{V} = 2\pi x \cdot h(x) \dd{x}$
\end{itemize}

\Note[Cylindrical Shell Method Formula]{
    Rotation about the $y$-axis (vertical shells):
    \[ 
        V = \int_{a}^{b} 2\pi x \cdot h(x) \dd{x}
    \]
    Rotation about the $x$-axis (horizontal shells):
    \[ 
        V = \int_{c}^{d} 2\pi y \cdot h(y) \dd{y}
    \]
    where $h$ is the height of the shell and the other variable is the radius.
}

\Example{
    Find the volume of the solid obtained by rotating the region bounded by $y = 2x^2 - x^3$ and $y = 0$ about the $y$-axis.
}{
    First, find where the curve intersects the $x$-axis: $2x^2 - x^3 = 0 \Rightarrow x^2(2-x) = 0$, so $x = 0$ or $x = 2$. \\
    Using the shell method with rotation about the $y$-axis:
    \begin{itemize}
        \item Radius: $r = x$
        \item Height: $h(x) = 2x^2 - x^3$
    \end{itemize}
    \begin{align*}
        V &= \int_{0}^{2} 2\pi x (2x^2 - x^3) \dd{x} \\
          &= 2\pi \int_{0}^{2} (2x^3 - x^4) \dd{x} \\
          &= 2\pi \left[ \frac{x^4}{2} - \frac{x^5}{5} \right]_{0}^{2} \\
          &= 2\pi \left( 8 - \frac{32}{5} \right) = \frac{16\pi}{5}
    \end{align*}
}

\Note[When to Use Shells vs. Disks/Washers]{
    \textbf{Use the Shell Method when:}
    \begin{itemize}
        \item Rotating about an axis parallel to the slices of the region
        \item The region is easier to describe with one function of $x$ (for rotation about $y$-axis) or $y$ (for rotation about $x$-axis)
        \item Using washers would require solving for the inverse function
    \end{itemize}
    \textbf{Use the Disk/Washer Method when:}
    \begin{itemize}
        \item Rotating about an axis perpendicular to the slices
        \item The region is naturally described with functions already in the right form
        \item You have clear outer and inner boundaries
    \end{itemize}
    \textbf{General Rule:} Both methods work for any problem, but one is usually simpler. If the axis of rotation is parallel to the representative rectangle, use shells. If perpendicular, use disks/washers.
}


%%%%%%%%%%%%%%%%
%  Arc Length  %
%%%%%%%%%%%%%%%%

\subsection{Arc Length}

Consider a curve $y=f(x)$. We want to find the length of the curve from $x=a$ to $x=b$. We can approximate the curve by a series of line segments. The length of each line segment is given by the Pythagorean theorem: \[ 
    \dd{s} = \sqrt{\dd{x}^2 + \dd{y}^2} = \sqrt{1 + \left( \dv{y}{x} \right)^2} \dd{x}
\]
The total length of the curve is given by the sum of the lengths of the line segments: \[ 
    L = \int \dd{s} = \int \sqrt{1 + \left( \dv{y}{x} \right)^2} \dd{x}
\]
Similarly, for a curve $x=h(y)$, the length of the curve from $y=c$ to $y=d$ is given by: \[ 
    L = \int \dd{s} = \int \sqrt{1 + \left( \dv{x}{y} \right)^2} \dd{y}
\]
\Note[Arc Length Formula]{
    \[ 
        L = \int \dd{s}
    \] where, 
    \begin{align*}
        ds &= \sqrt{1 + \left( \dv{y}{x} \right)^2} \dd{x} \quad \text{for } y = f(x), a \leq x \leq b \\
        ds &= \sqrt{1 + \left( \dv{x}{y} \right)^2} \dd{y} \quad \text{for } x = h(y), c \leq y \leq d
    \end{align*}
}


%%%%%%%%%%%%%%%%%%
%  Surface Area  %
%%%%%%%%%%%%%%%%%%

\subsection{Surface Area}

Consider a curve $y=f(x)$ rotated about the $x$-axis. We want to find the surface area of the resulting surface. We can approximate the surface by a series of frustums. The surface area of each frustum is given by: \[ 
    \dd{S} = 2\pi y \dd{s} = 2\pi f(x) \sqrt{1 + \left( \dv{y}{x} \right)^2} \dd{x}
\]
The total surface area of the surface is given by the sum of the surface areas of the frustums: \[ 
    S = \int \dd{S} = \int 2\pi y \dd{s} = \int 2\pi f(x) \sqrt{1 + \left( \dv{y}{x} \right)^2} \dd{x}
\]
Similarly, for a curve $x=h(y)$ rotated about the $y$-axis, the surface area of the resulting surface is given by: \[ 
    A = \int 2\pi x \dd{s} = \int 2\pi h(y) \sqrt{1 + \left( \dv{x}{y} \right)^2} \dd{y}
\]

\Note[Surface Area Formula]{
    \begin{align*}
        S &= \int \dd{S} \\
          &= \int 2 \pi y \dd{s} \quad \text{rotation about } x \text{--axis} \\
          &= \int 2 \pi x \dd{s} \quad \text{rotation about } y \text{--axis}
    \end{align*}
    where, 
    \begin{align*}
        dS &= 2\pi y \dd{s} = 2\pi f(x) \sqrt{1 + \left( \dv{y}{x} \right)^2} \dd{x} \quad \text{for } y = f(x), a \leq x \leq b \\
        dS &= 2\pi x \dd{s} = 2\pi h(y) \sqrt{1 + \left( \dv{x}{y} \right)^2} \dd{y} \quad \text{for } x = h(y), c \leq y \leq d
    \end{align*}
}


%%%%%%%%%%%%%%%%%%%%
%  Center of Mass  %
%%%%%%%%%%%%%%%%%%%%

\subsection{Center of Mass}

Suppose we want to find the center of mass of a region bounded by two curves $f(x)$ and $g(x)$ on the interval $[a,b]$. \\ 
The mass is   \[ 
      M = \rho \int_{a}^{b} \left( f(x) - g(x) \right) \dd{x}
\]
Next, we need the \textbf{moments} of the region. There are two moments:
\begin{align*}
    M_x &= \rho \int_{a}^{b} \frac{1}{2} \left[ f(x)^2 - g(x)^2 \right] \dd{x} \\
    M_y &= \rho \int_{a}^{b} x \left[ f(x) - g(x) \right] \dd{x}
\end{align*}
The coordinates of the center of mass, $(\bar{x}, \bar{y})$, are given by:
\Note[Center of Mass Formula]{
    \[
        \bar{x} = \frac{M_y}{M} = \frac{1}{A} \int_{a}^{b} x \left[ f(x) - g(x) \right] \dd{x}
    \]
    \[
        \bar{y} = \frac{M_x}{M} = \frac{1}{A} \int_{a}^{b} \frac{1}{2} \left[ f(x)^2 - g(x)^2 \right] \dd{x}
    \]
    where,   \[ 
          A = \int_{a}^{b} \left[ f(x) - g(x) \right] \dd{x}
    \]
}


%%%%%%%%%%%%%%%%%
%  Probability  %
%%%%%%%%%%%%%%%%%

\subsection{Probability}

Every continuous random variable $X$, has a probability density function $f(x)$. Probability density functions satisfy the following conditions:
\begin{enumerate}
    \item $f(x) \geq 0$ for all $x$. 
    \item $\displaystyle \int_{-\infty}^{\infty} f(x) \dd{x} = 1$
\end{enumerate}

Probability density finctions can be used to determine the probability that a continuous random variable lies between two values, say $a$ and $b$. This probability is denoted by $P(a \leq X \leq b)$. 
\Note{
    \[ P(a \leq X \leq b) = \int_{a}^{b} f(x) \dd{x} \]
}

Probability density functions can also be used to determine the mean of a continuous random variable. The mean is given by:
\Note[Mean of a Continuous Random Variable]{
    \[ \mu = \int_{-\infty}^{\infty} x f(x) \dd{x} \]
}
