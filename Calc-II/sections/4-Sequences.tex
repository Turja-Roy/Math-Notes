\section{Sequences}

%%%%%%%%%%%%%%%%
%  Definition  %
%%%%%%%%%%%%%%%%

\subsection{Definition}

\Definition{Sequence}{
    A \textbf{sequence} is a function whose domain is the set of natural numbers $\mathbb{N}$. The function is denoted by $a_n$ and the value of the function at $n$ is denoted by $a_n$. \\ 
    Various ways of representing a sequence are:
    \[ \{a_1, a_2, \dots, a_n, a_{n+1}, \dots\} \qquad \{a_n\} \qquad \{a_n\}_{n=1}^{\infty} \]
}


%%%%%%%%%%%%%%%%%%%%%%%%%%%%%%%%%
%  Precise Definition of Limit  %
%%%%%%%%%%%%%%%%%%%%%%%%%%%%%%%%%

\subsection{Precise Definition of Limit of a Sequence}

\Note[Precise Definition of Limit]{
    \begin{enumerate}
        \item We say that \[ 
                \lim_{n \to \infty} a_n = L
            \] if \[ 
                \forall \epsilon > 0, \exists N \in \N : \forall n \geq N, |a_n - L| < \epsilon
             \]
        \item We say that \[ 
                \lim_{n \to \infty} a_n = \infty 
            \] if \[ 
                \forall M \in \R, \exists N \in \N : \forall n \geq N, a_n > M
             \]
        \item We say that \[ 
                \lim_{n \to \infty} a_n = -\infty
            \] if \[ 
                \forall M \in \R, \exists N \in \N : \forall n \geq N, a_n < M
             \]
    \end{enumerate}
}


%%%%%%%%%%%%%%%%%%%%%%%%%%%%%%
%  Convergence of Sequences  %
%%%%%%%%%%%%%%%%%%%%%%%%%%%%%%

\subsection{Convergence of Sequences}
\Theorem{Convergence of Sequences}{
    A sequence $\left\{ a_n \right\}$ is said to be \textbf{convergent} if there exists a real number $L$ such that \[ 
        \lim_{n \to \infty} a_n = L
    \]
}

\Theorem{}{
    Given the sequence $\left\{ a_n \right\}$ if we have a function $f(x)$ such that $f(n) = a_n$ and $\displaystyle \lim_{x \to \infty} f(x) = L$ then $\displaystyle \lim_{n \to \infty} a_n = L$.
}

\Note[Properties of Convergent Sequences]{
    If $\left\{ a_n \right\}$ and $\left\{ b_n \right\}$ are both convergent sequences, then:
    \begin{enumerate}
        \item
            \[
                \lim_{n \to \infty} (a_n \pm b_n) = \lim_{n \to \infty} a_n \pm \lim_{n \to \infty} b_n
            \]
        \item
            \[
                \lim_{n \to \infty} ca_n = c \lim_{n \to oo} a_n
            \]
        \item
            \[
                \lim_{n \to \infty} (a_n b_n) = \left( \lim_{n \to \infty} a_n \right)\left( \lim_{n \to \infty} b_n \right)
            \]
        \item
            \[
                \lim_{n \to \infty} \frac{a_n}{b_n} = \dfrac{\lim_{n \to \infty} a_n}{\lim_{n \to \infty} b_n}
                , \qquad \text{ provided } \lim_{n \to \infty} b_n \neq 0
            \]
        \item
            \[
                \lim_{n \to \infty} a_n^p = \left[ \lim_{n \to \infty} a_n \right]^p
                , \qquad \text{ provided } a_n \geq 0
            \]
    \end{enumerate}
}

\Theorem{Squeeze Theorem for Sequences}{
    If $\left\{ a_n \right\}$, $\left\{ b_n \right\}$ and $\left\{ c_n \right\}$ are sequences such that $a_n \leq b_n \leq c_n$ for all $n \geq N$ and $\displaystyle \lim_{n \to \infty} a_n = \lim_{n \to \infty} c_n = L$ then $\displaystyle \lim_{n \to \infty} b_n = L$.
}

\Theorem{}{
    \[ \text{If} \lim_{n \to \infty} \lvert a_n \rvert = 0, \text{ then } \lim_{n \to \infty} a_n = 0 \]
}

\underline{\textbf{Proof:}} \\ 
The main thing to this proof is to note that, \[ 
    -\lvert a_n \rvert \leq a_n \leq \lvert a_n \rvert
\]
Then note that, \[ 
    \lim_{n \to \infty} -\lvert a_n \rvert = -\lim_{n \to \infty} \lvert a_n \rvert = 0
\]
We then have that, \[ 
    0 \leq \lim_{n \to \infty} a_n \leq 0 
\] and so by the Squeeze Theorem, \[ 
    \lim_{n \to \infty} a_n = 0 
\]

\Theorem{}{
    The sequence $\left\{ r^n \right\}_{n=0}^{\infty}$ converges if $-1 < r \leq 1$ and diverges for all other values of $r$. Also, \[
        \lim_{n \to \infty} r^n
        \begin{cases}
            0, &\text{ if } -1 < r < 1 \\
            1, &\text{ if } r = 1
        \end{cases}
     \]
}

\Theorem{}{
    For the sequence $\left\{ a_n \right\}$ if both $\displaystyle \lim_{n \to \infty} a_{2n} = L$ and $\displaystyle \lim_{n \to \infty} a_{2n+1} = L$, then $\left\{ a_n \right\}$ is convergent and $\displaystyle \lim_{n \to \infty} a_n = L$.
}

\underline{\textbf{Proof:}}  \\ 
Let $\epsilon > 0$. Then since $\displaystyle \lim_{n \to \infty} a_{2n} = L$, \[ 
    \exists N_1 \in \N : \forall n \geq N_1, \lvert a_{2n} - L \rvert < \epsilon
\]
Similarly, because $\displaystyle \lim_{n \to \infty} a_{2n+1} = L$, \[ 
    \exists N_2 \in \N : \forall n \geq N_2, \lvert a_{2n+1} - L \rvert < \epsilon
\]
Now, let $N = \max\{2N_1, 2N_2+1\}$ and let $n > N$. Then either $a_n = a_{2k}$ for some $k > N_1$ or $a_n = a_{2k+1}$ for some $k > N_2$, And so in either case we have that \[ 
    \lvert a_n - L \rvert < \epsilon
\]
Therefore, $\displaystyle \lim_{n \to \infty} a_n = L$ and so $\left\{ a_n \right\}$ is convergent.


%%%%%%%%%%%%%%%%%%%%%%%%%%%%%%%%%%%%%
%  Bounded and Monotonic Sequences  %
%%%%%%%%%%%%%%%%%%%%%%%%%%%%%%%%%%%%%

\subsection{Bounded and Monotonic Sequences}

\Definition{Bounded Sequence}{
    A sequence $\left\{ a_n \right\}$ is \textbf{bounded} if \[ 
        \exists M \in \R : \forall n \in \N, \lvert a_n \rvert \leq M
    \]
}

\Note[Upper and Lower Bounds]{
    If \[
        \forall n, \exists m : m \leq a_n
    \] the sequence $\left\{ a_n \right\}$ is said to be \textbf{bounded below} and $m$ is a \textbf{lower bound} of the sequence. Similarly, if \[
        \forall n, \exists M : M \geq a_n
    \] the sequence $\left\{ a_n \right\}$ is said to be \textbf{bounded above} and $M$ is an \textbf{upper bound} of the sequence.
}

\Theorem{Bounded Sequence}{
    If a sequence $\left\{ a_n \right\}$ is both bounded above and below, then it is bounded. That is, if \[ 
        \exists m, M \in \R : \forall n \in \N, m \leq a_n \leq M
    \] then $\left\{ a_n \right\}$ is bounded.
}

\Definition{Monotonic Sequence}{
    A sequence $\left\{ a_n \right\}$ is \textbf{monotonic} if for all $n \in \N$ \[ 
        a_{n+1} \geq a_n \quad \text{or} \quad a_{n+1} \leq a_n
    \]
}

\Theorem{}{
    If a sequence $\left\{ a_n \right\}$ is both bounded and monotonic, then it is convergent.
}
